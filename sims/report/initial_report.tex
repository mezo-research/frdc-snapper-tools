% Initial writeup for the snapper project; started 12/17/2014 by Cole M.

\documentclass{article}
\usepackage[]{graphicx}
\usepackage[margin=1in]{geometry}
\usepackage{float}

\begin{document}
\title{Simulation testing the red snapper stock assessment:\\ Update on the
  status of the operating and estimation models}
\author{Cole Monnahan}
\date{\today{}}
\maketitle

This document briefly describes the initial stage of creating a simulation
framework for the red snapper assessment model. I briefly outline the
assessment, the software framework adopted for the simulation, and the
process of creating the simulation models.

The assessment model (AM) is an age-structured statistical catch-at-age
model written in Stock Synthesis (SS3). The model takes data collected from
the fishery and survey and estimates the underlying population dynamics,
including both biological and fishery properties of the stock. The original
assessment model runs from 1978 to 2013 with one season, 2 sexes, 6 fishing
fleets and 1 survey (YOY survey). Natural mortality is fixed at 0.2, and
growth is the standard von Bertalanfy with parameters fixed for females but
offsets estimated for males. Stock recruitment is Beverton-Holt with
steepness fixed at 0.8 but $R_0$ estimated. Length-based selectivity is
used for all fleets, with varying functional forms (logistic, dome-shaped,
etc.). The model is fitted to three types of data: indices of abundance,
length compositions, and conditional age-at-length compositions. CPUE from
fleets run from approximately 1978-2013 and a index from the survey from
1993-2013. There are length compositions for the fleets but not the
survey. Instead of traditional age compositions, the assessment uses
conditional age-at-length data. This type of data is created by taking
lengths and ages of fish and then creating an age distribution for each
length bin (e.g.\ 10-12cm) for each year/fleet.

The properties of the fishery, such as selectivity and effort, are mostly
important because of their impact on the estimation process (they obscure
our ``view'' of the biology).. The biological properties (productivity,
growth, reproduction, etc.) are considered to be inherent to the stock, and
thus would not change under different fishing schemes. As such, for
simulation purposes, it is the biological properties that are most
important and the complexity of the fishery can likely be simplified down
without losing the core biological properties.

The snapper simulation framework will be conducted in \texttt{ss3sim}, an R
package that facilitates rapid, flexible, and reproducible simulation
testing of stock assessment models using SS3. R is used for the entire
simulation process from specifying and running the simulations to reading,
processing and exploring the results. A \texttt{ss3sim} study consists an
operating model (OM) used to generate a true underlying population dynamics
from which data are generated, and an estimation model (EM) used to
estimate the underlying population dynamics from the generated
data. \texttt{ss3sim} uses R functions to manipulate the OM (e.g.\ fishing
patterns, process error, time-varying trends in parameters, etc.), as well
as the EM (data types, turn on/off estimation of parameters, etc.), making
it easy to induce structural differences between models. The analyst can
thus easily manipulate the models to create combinations of different
scenarios for which the performance of a model can be compared.

The OM is derived from the AM, but is not a strict copy for several
reasons. First, some aspects of the model inherently need to be changed to
be usable as an operating model. For example recruitment deviations need to
be inputted directly in the .par file. Second, the \texttt{ss3sim} package
requires the model to have a certain structure to be usable. The myriad
features and conditional options available to SS3 make it challenging to
write generic functions to interact with the model. As such, certain
features are turned off, such as 2 sexes. These features are not inherently
impossible in the \texttt{ss3sim} framework but are simply not yet
supported. Thus, currently unsupported features could be added for this
specific project. Here is a cursory list of some of the more impactful
changes made to create the OM:
\begin{enumerate}
\item The differences in males were removed, reducing the model to one
  sex. This affects mortality and growth, which had estimated offsets for
  males, but also the data which needed to be inputted by sex. Removing
  male catches changed the catch patterns in some fleets, since some
  (apparently) caught more of one sex in different years. (Still need to
  investigate this).
\item For simplicity in this first phase I removed all but the first fleet
  and the survey. These can be added back to the model if needed at a later
  stage.
\item The conditional age-at-length data was replaced with standard age
  composition data, since the former data type is not currently supported
  by \texttt{ss3sim}, although it will be in the next 6 months. For now
  \texttt{ss3sim} samples from true age and length compositions
  independently and with out bias, although overdispersion is an
  option. The next version will contain much more realistic and richer
  options for supported data. (I am helping develop this part of the
  package right now, for another project).
\end{enumerate}

The OM has been created, tested, and compared against the original AM. The
OM is not a valid model in that it will have a meaningless likelihood value
and would not be estimable. It is simply run for a single iteration, using
the parameters specified by the analyst, to generate expected values of the
population dynamics which can then be sampled to create ``data'' for use
in the EM.

The EM is a copy of the OM with estimation turned back on. Once realistic
data are then passed to it, its likelihood will be meaningful and the
parameters should be estimable. In their base form, the OM and EM have
identical biological and fishery structure -- the only difference is on the
statistical side. However, the analyst can readily alter either model to
create differences. The EM has also been created and tested under some very
basic scenarios. The first step is to pass very little recruitment
deviation error (process error in SS3) to the OM, and extremely informative
data to the EM to ensure that estimates are unbiased and relatively
small. The plot folder has some examples of these tests, using a
exploitation pattern similar to the fishery (but not exact), and very small
recruitment deviations under two data scenarios: ``D100'' has age and
length comps every 2 years with high sample sizes, and ``D101'' has comps
every 4 years with lower sample sizes. We expect both to be unbiased but
D101 to have larger relative errors.


\end{document}
